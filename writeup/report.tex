\documentclass[onecolumn]{aastex62}

\shorttitle{photon counting \& detectors}
\shortauthors{j. birky}


\begin{document}

\title{\sc Lab 1: Photon Counting \& Detectors}

\author{Jessica Birky (A13002163)}

\begin{abstract}

This example manuscript is intended to serve as a tutorial and template for
authors to use when writing their own AAS Journal articles. The manuscript
includes a history of and documents the new features in the
previous versions as well as the new features in version 6.2. This
manuscript includes many figure and table examples to illustrate these new
features.  Information on features not explicitly mentioned in the article
can be viewed in the manuscript comments or more extensive online
documentation. Authors are welcome replace the text, tables, figures, and
bibliography with their own and submit the resulting manuscript to the AAS
Journals peer review system.  The first lesson in the tutorial is to remind
authors that the AAS Journals, the Astrophysical Journal (ApJ), the
Astrophysical Journal Letters (ApJL), and Astronomical Journal (AJ), all
have a 250 word limit for the abstract.  If you exceed this length the
Editorial office will ask you to shorten it.

\end{abstract}
\bigskip

\section{Introduction} 
In order to understand the. Advancements in technology in the 20th century has now made it possible to observe electromagnetic radiation from the scales of $10^{-12}-10^3$m (as short as gamma rays to as long as radio waves). Understanding how detectors work, the statistics of light collection, as well as the limitations of the instrument and sources of contaminations are a fundamental first steps before performing any scientific analysis with an instrument.

CCDs, the photoelectric effect

% ==================================
\section{Observations}
Nickel telescope, what are bias and flat frames


% ==================================
\section{Data Reduction \& Methods}
Poisson statistics, procedure to combine frames and subtract bias, how to calclate  point to code

% ==================================
\section{Data Analysis \& Modeling}
\subsection{Bias Frames}

\subsection{Combined Images \& Histograms}
The figures in \ref{fig:flats1} and \ref{fig:flats2} show the combined, bias-subtracted histograms and images for each of the eight different exposure times (3, 6, 12, 24, 48, 96, 192, 384, and 768 seconds). The left side of each panel shows the histogram of the number of detector counts in analog to digital units (ADU), normalized such that the sum of all of the bins is one (plotted in black). The solid blue vertical line marks the mean of the data, and the dashed blue line marks the median, indicating which distrubtions are skewed by outlying points (where the mean is far from the median): in the combined bias frame the mean is skewed higher than the center of the distribution, and for all of the other exposures the mean is skewed lower than the center. The two shades of green mark $\pm1$ and $\pm2$ standard deviations above and below the mean. For all distributions, the $\pm1\sigma$ lines lie far outside of the bulk of the distribution around the median, which indicates again that there are many outlying the part of the distribution shown. Finally, the distribution in red shows the expected Poisson distribution, as a function of the mean of the data ($\bar{x}$), with a one standard deviation range also shaded in red (where the expected standard deviation for the Poisson is $\sigma=\sqrt{\bar{x}}$).  

The right side of each panel shows the 2D images. In order to visualize the detector counts per pixel, we must map the number of counts to an RGB color value, which we use the matplolib 'gray' colormap for. To emphasize the features in the data, we choose the min an max values of the colormap to be the 10th and 90th percentile counts of the data.


\begin{figure}[ht]
\plotone{plots/exposure0.png}
\caption{Combined bias frames.} \label{fig:bias}
\end{figure}
\begin{figure}[ht]
\plotone{plots/exposure3.png}
\plotone{plots/exposure6.png}
\plotone{plots/exposure12.png}
\plotone{plots/exposure48.png}
\caption{Combined, bias-subtracted, flat frame histograms and images for varying exposure times (0, 3, 6, and 12 sec). \label{fig:flats1}}
\end{figure}
\begin{figure}[ht]
\plotone{plots/exposure96.png}
\plotone{plots/exposure192.png}
\plotone{plots/exposure384.png}
\plotone{plots/exposure768.png}
\caption{Combined, bias-subtracted, flat frame histograms and images for varying exposure times (96, 192, 384, and 768 sec). \label{fig:flats2}}
\end{figure}

\begin{figure}[ht]
\plottwo{plots/mean_vs_variance.png}{plots/exptime_vs_meancount.png}
\caption{Mean vs. variance for all bias-subtracted exposures.} \label{fig:mean_var}
\end{figure}

% ==================================
\section{Discussion}
\subsection{Poisson Comparisons}

\subsection{Sources of Noise}
Variations in the quantum efficiency of each pixel

\subsection{Systematic Effects}
Checked for systematics due to temperature variation, and checked for differences in different datasets. Found temperature to vary only within 4C

% ==================================
\section{Conclusion}

% ==================================
\section{Author Contributions}

This project was done in collaboration with Julian Beas-Gonzalez and Russell Van-Linge (Group E), using data collected by Group B. The code written is my own (except for a few snippets for making histograms and reading fits files from the class notes), and can be found on github: \href{https://github.com/jbirky/photonCounting}{https://github.com/jbirky/photonCounting}.

% ==================================
\section{Appendix}

\subsection{Statistics}
Mean and variance for a set of data points $x={x_1, ...,x_N}$:
\begin{equation}
	\bar{x} = \frac{1}{N} \sum^N_{i=1} x_i  
\end{equation}
\begin{equation}
	s^2 = \frac{1}{N-1} \sum^N_{i=1} (x_i - \bar{x})^2
\end{equation}

\subsection{Code}


% \begin{thebibliography}{}
% \bibitem[Astropy Collaboration et al.(2013)]{2013A&A...558A..33A} Astropy Collaboration, Robitaille, T.~P., Tollerud, E.~J., et al.\ 2013, \aap, 558, A33 
% \end{thebibliography}


\end{document}

